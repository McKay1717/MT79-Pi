\documentclass[a4paper,10pt]{report}
\usepackage[utf8]{inputenc}
\usepackage{amsmath}

% Title Page
\title{Approximation du nombre $$\pi$$}
\author{Nicolas Iung, Aurélien Blais}



\begin{document}
\maketitle

\chapter{Méthode des séries}
\section{Somme des inverses des carrés}

\subsection{Question 1}
\subsubsection{Montrer que $a_0$ = 0}
On sait d'après l'énoncé que sur l'intervalle $]0, \pi[$, $f(t) = 1$ et que sur l'intervalle $]-\pi, 0[$, $f(t) = -1$\\

\begin{align*}
a_0 &= \frac{1}{2\pi} \int_{0}^{2\pi} f(t)dt \\
a_0 &= \frac{1}{2\pi} (\int_{0}^{\pi} 1 dt + \int_{\pi}^{2\pi} -1 dt)\\
a_0 &= \frac{1}{2\pi} (\pi - \pi)\\
a_0 &= 0
\end{align*}

\subsubsection{Montrer que $a_n$ = 0}
\begin{align*}
a_n &= \frac{2}{2\pi} \int_{0}^{2\pi} f(t)cos(nt) dt \\
a_n &= \frac{2}{2\pi} (\int_{0}^{\pi} cos(nt) dt + \int_{\pi}^{2\pi} -cos(nt) dt)\\
a_n &= \frac{2}{2\pi} (\frac{sin(n\pi)}{n} - \frac{sin(n\pi)}{n} )\\
a_n &= 0
\end{align*}

\subsubsection{Montrer que $b_n = \frac{4}{2\pi}(\frac{1-cos(n\pi)}{n})$}
\begin{align*}
b_n &= \frac{2}{2\pi} \int_{0}^{2\pi} f(t) sin(nt) dt\\
b_n &= \frac{2}{2\pi} (\int_{0}^{\pi} sin(nt) dt + \int_{\pi}^{2\pi} -sin(nt) dt)\\
b_n &= \frac{2}{2\pi} [(\frac{1}{n} - \frac{cos(n\pi)}{n}) + \frac{cos(2n\pi) - cos(n\pi)}{n}]\\
b_n &= \frac{2}{2\pi} [\frac{1}{n} (-cos(n\pi) + 1 + cos(2n\pi) - cos(n\pi))]\\
b_n &= \frac{2}{2\pi} [\frac{1}{n} (-2cos(n\pi) + 2)]\\
b_n &= \frac{2}{2\pi} [\frac{2}{n} (-cos(n\pi) + 1)]\\
b_n &= \frac{4}{2\pi} [\frac{1 - cos(n\pi)}{n}]
\end{align*}

\end{document}          
